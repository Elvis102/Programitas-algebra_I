% Criba de Eratóstenes
% Ejemplo de como usar código en lua para generar un documento tex
% Procesar con lualatex

\documentclass{beamer}

\usepackage{pdfpages}
\mode<presentation>
{ \usetheme{Madrid} }

\usepackage[spanish]{babel}
\usepackage{xcolor}
\usepackage{luacode}
\usepackage{url}
\usepackage{amsmath}

\newcommand{\hlightP}[1]{%
  \ooalign{\hss\makebox[0pt]{\fcolorbox{red!30}{green!40}{$#1$}}\hss\cr\phantom{$#1$}}%
}
\newcommand{\hlightC}[1]{%
  \ooalign{\hss\makebox[0pt]{\fcolorbox{green!30}{red!40}{$#1$}}\hss\cr\phantom{$#1$}}%
}

\begin{document}

% http://wiki.luatex.org/index.php/Writing_Lua_in_TeX

%https://wiki.contextgarden.net/Programming_in_LuaTeX

\title{La Criba de Eratóstenes}
\author{Pablo De Nápoli}
\maketitle

\begin{frame}{La Criba de Eratóstenes}

 \begin{itemize}
 
 \item   La \alert{criba de Eratóstenes} (matemático griego, 276--194 AC) es un algoritmo que permite hallar todos los números primos menores que un 
 número natural $n$ dado. 
\item Se forma una tabla con todos los números naturales comprendidos entre 2 y $n$, 
 y se van tachando los números que no son primos de la siguiente manera: Comenzando por el 2, se tachan todos 
 sus múltiplos; comenzando de nuevo, cuando se encuentra un número entero que no ha sido tachado, ese número es 
 declarado \alert{primo}, y se procede a tachar todos sus múltiplos, así sucesivamente. 
 
\item El proceso termina cuando el cuadrado del siguiente número confirmado como primo es mayor que n.

\end{itemize}

(fuente: Wikipedia). Veremos como ejemplo como encontrar todos los primos menores que 100.

\end{frame}

\begin{luacode}
function mostrar_criba(n,criba,cartel)
    tex.print("\\begin{frame}{",cartel,"}")
    columnas=10
    tex.print("$$")
    tex.print("\\begin{pmatrix}")  
    for i = 1, n do 
        if criba[i] then
            tex.sprint("\\hlightP")   
        else
            tex.sprint("\\hlightC")      
        end
    tex.sprint("{")
    tex.sprint(i)
    tex.sprint("}")
    if i % columnas== 0 then 
      tex.sprint(" \\\\")
    else 
      tex.sprint(" & ")
    end
  end
tex.print("\\end{pmatrix}")  
tex.print("$$")
tex.print("\\end{frame}")
end 

--- n representa hasta donde queremos calcular la tabla de primos

local n = 100
local criba = {}
local primos = {}

--- incialemnte ningún número fue tachado en la criba
criba[1]=false
for i = 2, n do criba[i] = true end

mostrar_criba(n,criba,"Criba inicial")
i=1
for i = 2, n do
 if criba[i] then
   -- encontramos un primo
   -- cuando el cuadrado del primo confirmado supera n, el algoritmo termina
   if i*i>n then break end
   -- sino tachamos sus múltiplos, son compuestos
   tachamos_alguno=false
   for j = i*2, n, i do 
      criba[j] = false
      tachamos_alguno=true 
    end
    if tachamos_alguno then
       mostrar_criba(n,criba,"tachamos los múltiplos de ".. tostring(i))
    end
 end
end

for i = 2, n do
 if criba[i] then
    primos[#primos+1] = i
  end     
end

tex.print("\\begin{frame}{Primos encontrados}")
    
tex.print("Los primos menores que ", n," son")    
-- Muestra la lista de primos en el documento latex
tex.sprint(table.concat(primos, " "))
tex.print("\\end{frame}")
\end{luacode}


\end{document}


